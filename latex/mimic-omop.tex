% Sample file for AES paper
\documentclass{aes2e}

% Metadata Information
\jyear{2010}
\jmonth{October}
\jvol{1}
\jnum{1}


\begin{document}

% Page heads
\markboth{A1LASTNAME AND A2LASTNAME}{MIMIC in the OMOP Common Data Model}


% Title portion
\title{MIMIC in the OMOP Common Data Model\thanks{To whom correspondence should be addressed Tel: +1-240-381-2383; Fax: +1-202-508-3799; e-mail: info@schtm.org}}

%Author Info.
\authorgroup{
\author{A1FIRSTNAME A1LASTNAME},
\role{AES Member}
AND \author{A2FIRSTNAME A2LASTNAME},
\role{AES Fellow}
\email{(abc@abc.com)\quad\quad\quad\quad\quad\quad\quad\quad\quad\quad\quad\quad (xyz@abc.com)}
\affil{Universityxyz, City, Country}
}

%Abstract
\abstract{%
This paper discusses the implementation of spectral delay using
filters comprising a cascade of many low-order allpass filters and an
equalizing filter. The spectral delay filters have chirp-like impulse
responses causing a large, frequency-dependent delay that is useful in
audio effects processing. An equalizing filter design and a multirate
technique, which stretches the allpass filters, impulse
response, are introduced.}


\maketitle

%Head 1
\section{INTRODUCTION}
% ICU: a paradox 
Intensive Care Units (ICUs) are particularly sensitive units  where demand of
care is rising\cite{angus2000} and mortality is up to 30\% which is a major
health care problem \cite{icu-mortality}. Studies have shown that intensivists
use a limited number of clinical information concepts to guide the
decision\cite{icu-evidence} and that the medical practices are sparse and
variable. Knowing that ICU patients' health record are highly detailed
including connected devices this is a paradox.

% Medical database hopes TODO/ syntetise
The increasing adoption of Electronic Health Records (EHR) systems worldwide
makes it possible to capture large amounts of clinical data
\cite{bigdata-promise} and big data mining has the potential to play an
important role in clinical medicine \cite{bigdata-mining}. Indeed on the basis
of various patient medical
informations (clinical, physiologic, genemomic, laboratory, imaging, reports,
environement) expectations are:
- minimize costs while improving the clinical outcomes of individuals and
  populations thanks to observational clinical research and real time
  algorithms
- Drug Adverse Event
- Drug/Drug Interaction
- Clinical research
- Personalized medicine 
- Medical Decision Guidance
- Early warning systems

% Existing ICU databses
Several commercial or noncommercial, opensource or nonopensource ICU databases
have been developed

\textbf{CUB-REA} is a semi-automatic collection of 25 years and over 300k ICU
patients stays from 30 distinct ICUs in Paris region and covers administrative
data only \cite{cubrea-descr,cubrea-website}. Yet it is subject of 15
international publication and access is under conditions.

\textbf{OutcomeREA} is a manual collection of 20 years over 20k UCU patients
stays with medium granularity data from 20 distinct ICUs in France. Yet it is
subject of 50 international publications and access is under conditions.

\textbf{eICU} \cite{eicu-website} is an automatic collection of 2 years of 1.5M
ICU patients stays from 180 hospitals in the USA and covers high granularity
data. It is a commercial database.

\textbf{MIMIC} (Medical Information Mart for Intensive Care) is a
semi-automatic collection of 10 years and over 60k ICU patients stays with very
high granularity (including EKG) from 1 ICU in Boston. Yet it is subject of 300
international publication. It is a freely-available database.

% Common Data Models
All those databases do have their own dedicated model. Their structural model
are all based on relational database but all do have tables and columns with
different meaning and granularity. As an example MIMIC do have two
inputevents tables reflecting its source center changed its EHR. Also their
conceptual model are mostly different. For example MIMIC do have ICD9 for
condition terminology, while french database CUBREA do have both CIM9 and
CIM10. A lot of research have been made on each of these databases
independently. While some studies have shown that results are not replicable
from one to another database \cite{omop-replicability} and that keeping the
local conceptual model \cite{fhir-deep} and structure \cite{imi-protect} of
database for research leads to better outcomes, a dozen of common data model
(CDM) have emerged.

% Choice of OMOP + Existing OMOP
\textbf{Observational Medical Outcomes Partnership Common Data Model} (OMOP) is
a CDM designed for multicentric Drug adverse Event and now enlarges to medical,
clinical and also genomic use cases. OMOP provides both structural (as as set
of relational tables) and conceptual (as a set of standard terminologies) such
SNOMED for diagnoses, RxNORM for drug ingredients and LOINC for laboratory
results. While OMOP has proven its fiability \cite{omop-eval} the fact that
concept mapping process is known to have impact on results
\cite{omop-concept-impact} and that applying the same protocol on different
data sources leads to different results \cite{omop-replicability} reveals the
importance of keeping the local codes to allow local analysis. Several example
of transforming databases into OMOP have been published
\cite{omop-german,omop-nashville} and yet OMOP stores 682 milion patients
records from all over the world\cite{omop-bigboy}. OMOP had 5 versions, and
prones its strong backward compatibility.

\textbf{i2b2/SHRINE} is a medical cohort discovery tool used in more than 200
hospitals over the world. SHRINE is one of the attempt to federate multiple
instances of i2b2. The i2b2 star schema has proven its high flexibility thanks
to the modular design of the fact tables allowing storing numerics, characters
or concepts. Its single terminology model is a path based hierarchical table
does not allow to modelise graph ontology (such snomed). While i2b2 is highly
efficient for cohort discovery, it's model wasn't designed for ad-hoc analysis.
The n*n terminology mapping initiated in SHRINE has been described time
consuming and inefficient.
	
\textbf{Fast Healthcare Interoperability Resources} (HL7-FHIR) is a medical
data exchange API specification. FHIR provides a structural CDM that can be
materialized as JSON, XML or RDF format. FHIR is flexible and does not specify
a standard conceptual model so that each hospital can add extension to
implement specific data or share within it's local terminology making each FHIR
implementation sensibly divergent. While some research show it as a promising
CDM for ad-hoc analysis \cite{fhir-google} or cohort discovery
\cite{fhir-paris}, its graph nature adds a layer of transformation making usage
complicated for data-scientists as well as difficult to create standardized
analysis. Finally the model envolves and does not make the asumption of
backward compatibilities along the versions.

%- PCORnet, the National Patient-Centered Clinical Research Network (http://pcornet.org/pcornet-common-data-model/)
%        - have been created to monitore the safety of FDA-regulated medical products.
%	- PCORnet Common Data Model (CDM) integrate multiple data from different sources and leverages standard terminologies and coding systems for healthcare (including ICD, SNOMED, CPT, HCPSC, and LOINC) to enable interoperability with and responsiveness to evolving data standards.
%	- The first version of the CDM was released in 2014, and there have been 3 major releases and one minor update since then (last release CDM v4.1: Released May 18, 2018 )

Among the other CDM, OMOP looks like the best fit as it allows both
multicentric standardised analysis as well as monocentric specific modeling and
analysis. Still some questions remains that we propose to answer such the
difficulty of transforming/maintaining an OMOP dataset from an existing one,
how well the initial dataset is integrated and how much data is lost in the
process, how clear and simple the model is to be queried simply and efficiently
by scientists, how well design it is to be enriched by collaborative work , and
finally in what extend OMOP can integrate and makes feedbacks to intensivists
in a realtime context.

However OMOP is very ambitious in the level of work preprocessing and
mapping work needed.
In this study we decided to evaluate
Compared to PCORnet CDM, OMOP (6) :
- performes best in the evaluation database criteria compared with the other
  models (and PCORnet in particularly) : completeness, integrity, flexibility,
  simplicity of integration, and implementability.
- seems to accommodate the broadest coverage of standard terminologies.
- provides more systematic analysis with analytic library and visualizing tools
  from OMOP community : ACHILLES
- provides easier SQL models 
  
FHIR:
- does specify a common structural model
- does not specify a common terminology model, for most of the attributes
- has the descendent of HL7, it primary goal is data sharing at low granularity
  (eg: patient, device level)
- implementation may vary substancialy from one to other instance
- XML and JSON are both not optimized in a computational or user friendly to
  make queries
- API on production EHR are not able to export large amount of data while some
  work are in the process (FHIR bulk export)
- transformation from FHIR dataset to datascientist ready to process dataset
  may be one ETL per instance

OMOP shares the advantages of all above models. It allows local analysis with
raw values, and local terminologies as it stores. It adds values by using a
simple and common structural model. It allows standard analysis when needed,
and makes possible to compare. However, question still are:
- how transforming real datasets to OMOP is complicated
- how much dataset lose information
- how performances are affected 
- how well OMOP handle ICU database specificities

We limited the candidate data models to those designed and used for clinical
researches, and those freely available in the public domains without
restrictions.
% Evaluation
Filtering an audio signal with an allpass filter does not usually have a major effect on the signal's timbre. The allpass filter does not change the frequency content of the signal, but only introduces a phase shift or delay. Audibility of the phase distortion caused by an allpass filter in a sound reproduction system has been a topic of many studies, see, e.g., \cite{DEK1}, \cite{DEK2}. In this paper, we investigate audio effects processing using high-order allpass filters that consist of many cascaded low-order allpass filters. These filters have long chirp-like impulse responses. When audio and music signals are processed with such a filter, remarkable changes are obtained that are similar to the spectral delay effect  \cite{DEK3}, \cite{DEK4}.

\section{CHIRP-LIKE IMPULSE RESPONSES AND GROUP DELAY}
Filtering an audio signal with an allpass filter does not usually have a major effect on the signal's timbre. The allpass filter does not change the frequency content of the signal, but only introduces a phase shift or delay. Audibility of the phase distortion caused by an allpass filter in a sound reproduction system has been a topic of many studies, see, e.g., \cite{DEK1}, \cite{DEK2}. In this paper, we investigate audio effects processing using high-order allpass filters that consist of many cascaded low-order allpass filters. These filters have long chirp-like impulse responses.  These filters have long chirp-like impulse responses.  These filters have long chirp-like impulse responses. When audio and music signals are processed with such a filter, remarkable changes are obtained that are similar to the spectral delay effect  \cite{DEK3}, \cite{DEK4}.

%Head 2
\subsection{Chirp-Like Impulse Responses and Group Delay}
Filtering an audio signal with an allpass filter does not usually have a major effect on the signal's timbre. The allpass filter does not change the frequency content of the signal, but only introduces a phase shift or delay. Audibility of the phase distortion caused by an allpass filter in a sound reproduction system has been a topic of many studies, see, e.g., \cite{DEK1}, \cite{DEK2}. In this paper, we investigate audio effects processing using high-order allpass filters that consist of many cascaded low-order allpass filters. These filters have long chirp-like impulse responses. When audio and music signals are processed with such a filter, remarkable changes are obtained that are similar to the spectral delay effect  \cite{DEK3}, \cite{DEK4}.

%Head 3
\subsubsection{Chirp-Like Impulse Responses and Group Delay}
Filtering an audio signal with an allpass filter does not usually have a major effect on the signal's timbre. The allpass filter does not change the frequency content of the signal, but only introduces a phase shift or delay. Audibility of the phase distortion caused by an allpass filter in a sound reproduction system has been a topic of many studies, see, e.g., \cite{DEK1}, \cite{DEK2}. In this paper, we investigate audio effects processing using high-order allpass filters that consist of many cascaded low-order allpass filters.  When audio and music signals are processed with such a filter, remarkable changes are obtained that are similar to the spectral delay effect  \cite{DEK3}, \cite{DEK4}.
%Equation
\begin{equation}
A(z) = \frac{{a_1  + z^{ - 1} }}{{1 + a_1 z^{ - 1} }},
\end{equation}



Filtering an audio signal with an allpass filter does not usually have a major effect on the signal's timbre. The allpass filter does not change the frequency content of the \nobreak signal, but only introduces a phase shift or delay.\footnote{This point is emphasized by Loewer, see esp. p. (610).} Audibility of the phase distortion caused by an allpass filter in a sound reproduction system has been a topic of many studies, see, e.g., \cite{DEK1}, \cite{DEK2}. In this paper, we investigate audio effects processing using high-order allpass filters that consist of many cascaded low-order allpass filters. These filters have long chirp-like impulse responses. When audio and music signals are processed with such a filter, remarkable changes are obtained that are similar to the spectral delay effect  \cite{DEK3}, \cite{DEK4}.
\begin{equation}
\tau _{\textrm{g,max}}  = \left\{ \begin{array}{l}
 \tau _\textrm{g} (0) = \frac{{1 - a_1 }}{{1 + a_1 }},\textrm{when }a_1  \le 0 \\[4pt]
 \tau _\textrm{g} (\pi ) = \frac{{1 + a_1 }}{{1 - a_1 }},\textrm{when }a_1  > 0. \\
 \end{array} \right.\end{equation}

Filtering an audio signal with an allpass filter does not usually have a major effect on the signal's timbre. The allpass filter does not change the frequency content of the signal, but only introduces a phase shift or delay. Audibility of the phase distortion caused by an allpass filter in a sound reproduction system has been a topic of many studies, see, e.g., \cite{DEK1}, \cite{DEK2}. In this paper, we investigate audio effects processing using high-order allpass filters that consist of many cascaded low-order allpass filters. These filters have long chirp-like impulse responses. When audio and music signals are processed with such a filter, remarkable changes are obtained that are similar to the spectral delay effect  \cite{DEK3}, \cite{DEK4}.
%Paragraph listing
\begin{paralist}
\item{}Filtering an audio signal with an allpass filter does not usually have a major effect on the signal's timbre. The allpass filter does not change the frequency content of the signal, but only introduces a phase shift or delay. 
\item{}Audibility of the phase distortion caused by an allpass filter in a sound reproduction system has been a topic of many studies, see, e.g., \cite{DEK1}, \cite{DEK2}. 
\item{}In this paper, we investigate audio effects processing using high-order allpass filters that consist of many cascaded low-order allpass filters. These filters have long chirp-like impulse responses. 
\item{}When audio and music signals are processed with such a filter, remarkable changes are obtained that are similar to the spectral delay effect  \cite{DEK3}, \cite{DEK4}.
\end{paralist}

Filtering an audio signal with an allpass filter does not usually have a major effect on the signal's timbre. The allpass filter does not change the frequency content of the signal, but only introduces a phase shift or delay. Audibility of the phase distortion caused by an allpass filter in a sound reproduction system has been a topic of many studies, see, e.g., \cite{DEK1}, \cite{DEK2}. In this paper, we investigate audio effects processing using high-order allpass filters that consist of many cascaded low-order allpass filters. These filters have long chirp-like impulse responses. When audio and music signals are processed with such a filter, remarkable changes are obtained that are similar to the spectral delay effect  \cite{DEK3}, \cite{DEK4}.
\begin{arabiclist}
\item{}Green--function determined experimentally and published.
\item{}Black--function determined using similarity searches and published.
\item{}Red--function determined using similarity searches and determined in this study.
\item{}Blue--O-antigen structure unknown. Function determined using similarity searches and proposed in this study.
\end{arabiclist}

%Table
\begin{table}
\tabcolsep8.1pt
\tbl{Active sites and allosteric sites of the GNE MNK enzyme}{%
\begin{tabular}{@{}lccc@{}}\toprule
Excerpt No.& Genre & Spatial Mode & Corrlation\\\colrule
 1 & Pop       & FB   & 94\%\\
 2 & Classical & FB   & 33\%\\
 3 & Jazz      & FF   & 76\%\\
 4 & Arabian   & FF   & 41\%\\
 5 & GNE       & H220 & 45\%\\
 6 & GNE       & H45  & 93\%\\
 7 & MNK       & G416 & 74\%\\
 8 & MNK       & D413 & 72\%\\
 9 & MNK       & R420 & 94\%\\
10 & MNK       & N516 & 91\%\\\botrule
\end{tabular}}
\begin{tabnote}
Note. This table does not include sentence enhancement statutes.  This table does not include sentence enhancement statutes.
\end{tabnote}
\end{table}

%Figure
\begin{figure}
\centering
\includegraphics{aes2e-mouse.eps}
\caption{The spectral delay filter consists of \textit{M} allpass filters and an equalization filter.}
\end{figure}


\begin{figure*}
\centering
\includegraphics[width=23pc]{aes2e-mouse.eps}
\caption{This paper is organized as follows. In Section 1, we discuss the group delay of a cascade of first-order allpass filters and its relation to the chirp-like impulse response of the spectral delay filter. Furthermore, a multirate method to stretch the impulse response of the spectral delay filter is proposed. Section 2 discusses the amplitude envelope of the impulse response and suggests a design method for the equalizing filter. Section 3 presents application examples using the spectral delay filter. Section 4 concludes this paper.}
\end{figure*}
Filtering an audio signal with an allpass filter does not usually have a major effect on the signal's timbre. The allpass filter does not change the frequency content of the signal, but only introduces a phase shift or delay.
\begin{extract}
Filtering an audio signal with an allpass filter does not usually have
a major effect on the signal's timbre. The allpass filter does not
change the frequency content of the signal, but only introduces a
phase shift or delay. Audibility of the phase distortion caused by an
allpass filter in a sound reproduction system has been a topic of many
studies, see, e.g., \cite{DEK1}, \cite{DEK2}. In this paper, we
investigate audio effects \nobreak processing using high-order allpass filters that consist of many cascaded low-order allpass filters. These filters have long chirp-like impulse responses. When audio and music signals are processed with such a filter, remarkable changes are obtained that are similar to the spectral delay effect  \cite{DEK3}, \cite{DEK4}.
\end{extract}
Filtering an audio signal with an allpass filter does not usually have a major effect on the signal's timbre. The allpass filter does not change the frequency content of the signal, but only introduces a phase shift or delay. 
\[
\tau _\textrm{g} (\omega ) =  - \frac{{d\phi (\omega )}}{{d\omega }}.
\]
Audibility of the phase distortion caused by an allpass filter in a sound reproduction system has been a topic of many studies, see, e.g., \cite{DEK1}, \cite{DEK2}. In this paper, we investigate audio effects processing using high-order allpass filters that consist of many cascaded low-order allpass filters. These filters have long chirp-like impulse responses. When audio and music signals are processed with such a filter, remarkable changes are obtained that are similar to the spectral delay effect.
\begin{alphalist}
\item{}Green--function determined experimentally and published.
\item{}Black--function determined using similarity searches and published.
\item{}Red--function determined using similarity searches and determined in this study.
\item{}Blue--O-antigen structure unknown. Function determined using similarity searches and proposed in this study.
\end{alphalist}
Filtering an audio signal with an allpass filter does not usually have a major effect on the signal's timbre. The allpass filter does not change the frequency content of the signal, but only introduces a phase shift or delay. Audibility of the phase distortion caused by an allpass filter in a sound reproduction system has been a topic of many studies, see, e.g., \cite{DEK1}, \cite{DEK2}. 
%Enunciations
\begin{example}
In this paper, we investigate audio effects processing using high-order allpass filters that consist of many cascaded low-order allpass filters. These filters have long chirp-like impulse responses. 
\end{example}
Filtering an audio signal with an allpass filter does not usually have a major effect on the signal's timbre. The allpass filter does not change the frequency content of the signal, but only introduces a phase shift or delay. Audibility of the phase distortion caused by an allpass filter in a sound reproduction system has been a topic of many studies.
\begin{bulletlist}
\item{}Green--function determined experimentally and published.
\item{}Black--function determined using similarity searches and published.
\item{}Red--function determined using similarity searches and determined in this study.
\item{}Blue--O-antigen structure unknown. Function determined using similarity searches and proposed in this study.
\end{bulletlist}
Filtering an audio signal with an allpass filter does not usually have a major effect on the signal's timbre. The allpass filter does not change the frequency content of the signal, but only introduces a phase shift or delay. Audibility of the phase distortion caused by an allpass filter in a sound reproduction system has been a topic of many studies, see, e.g., \cite{DEK1}, \cite{DEK2}. In this paper, we investigate audio effects processing using high-order allpass filters that consist of many cascaded low-order allpass filters. These filters have long chirp-like impulse responses. When audio and music signals are processed with such a filter, remarkable changes are obtained that are similar to the spectral delay effect  \cite{DEK3}, \cite{DEK4}.
\begin{unnumlist}
\item{}Green--function determined experimentally and published.
\item{}Black--function determined using similarity searches and published.
\item{}Red--function determined using similarity searches and determined in this study.
\item{}Blue--O-antigen structure unknown. Function determined using similarity searches and proposed in this study.
\end{unnumlist}
Filtering an audio signal with an allpass filter does not usually have a major effect on the signal's timbre. The allpass filter does not change the frequency content of the signal, but only introduces a phase shift or delay. Audibility of the phase distortion caused by an allpass filter in a sound reproduction system has been a topic of many studies, see, e.g., \cite{DEK1}, \cite{DEK2}. In this paper, we investigate audio effects processing using high-order allpass filters that consist of many cascaded low-order allpass filters. These filters have long chirp-like impulse responses. When audio and music signals are processed with such a filter, remarkable changes are obtained that are similar to the spectral delay effect  \cite{DEK3}, \cite{DEK4}.

\section{SUMMARY}
Filtering an audio signal with an allpass filter does not usually have a major effect on the signal's timbre. The allpass filter does not change the frequency content of the signal, but only introduces a phase shift or delay. Audibility of the phase distortion caused by an allpass filter in a sound reproduction system has been a topic of many studies, see, e.g., \cite{DEK1}, \cite{DEK2}. In this paper, we investigate audio effects processing using high-order allpass filters that consist of many cascaded low-order allpass filters. These filters have long chirp-like impulse responses. When audio and music signals are processed with such a filter, remarkable changes are obtained that are similar to the spectral delay effect  \cite{DEK3}, \cite{DEK4}.

\section{CONCLUSION}
Filtering an audio signal with an allpass filter does not usually have a major effect on the signal's timbre. The allpass filter does not change the frequency content of the signal, but only introduces a phase shift or delay. Audibility of the phase distortion caused by an allpass filter in a sound reproduction system has been a topic of many studies, see, e.g., \cite{DEK1}, \cite{DEK2}. In this paper, we investigate audio effects processing using high-order allpass filters that consist of many cascaded low-order allpass filters. These filters have long chirp-like impulse responses. When audio and music signals are processed with such a filter, remarkable changes are obtained that are similar to the spectral delay effect  \cite{DEK3}, \cite{DEK4}. Note that articles might have a digital object identifier~\cite{DEK5}.

\section{ACKNOWLEDGMENT}
This research was conducted in fall 2008 when Vesa V\"alim\"aki was a visiting scholar at CCRMA, Stanford University. His visit was financed by the Academy of Finland (project no. 126310). The authors would like to Dr. Henri Penttinen for his comments and for the snare drum sample used in this work.

\bibliography{aes2e.bib}
\bibliographystyle{aes2e.bst}

% NOTE:
% - in case you are not using bibTex you have to manually edit the bibliograpy as below.
% - if submitting a bibTex file is not allowed you can copy the content from the aes2e.bbl file  
%\begin{thebibliography}{99}
%
%\newcommand{\enquote}[1]{``#1''}
%\providecommand{\url}[1]{\texttt{#1}}
%\providecommand{\urlprefix}{URL }
%\expandafter\ifx\csname urlstyle\endcsname\relax
%  \providecommand{\doi}[1]{[Online]. Available: \discretionary{}{}{}#1}\else
%  \providecommand{\doi}{doi:\discretionary{}{}{}\begingroup
%  \urlstyle{rm}\Url}\fi
%
%\bibitem{DEK1}
%D.~Preis, \enquote{Phase Distortion and Phase Equalization in Audio Signal
%  Processing---A Tutorial Review,} \emph{J. Audio Eng. Soc.}, vol.~30, no.~11,
%  pp. 774--779 (1982 Nov.).
%
%\bibitem{DEK2}
%J.~S. Abel, D.~P. Berners, \enquote{MUS424/EE367D: Signal Processing Techniques
%  for Digital Audio Effects,}  (2005), unpublished Course Notes, CCRMA,
%  Stanford University, Stanford, CA.
%
%\bibitem{DEK3}
%C.~Roads, \enquote{Musical Sound Transformation by Convolution,} presented at
%  the \emph{Int. Computer Music Conf.}, pp. 102--109 (1993).
%
%\bibitem{DEK4}
%C.~Roads, \emph{The Computer Music Tutorial} (MIT Press, Cambridge, MA), 1st
%  ed. (1996).
%
%\bibitem{DEK5}
%H.~Morgenstern, B.~Rafaely, \enquote{Spatial Reverberation and Dereverberation
%  Using an Acoustic Multiple-Input Multiple-Output System,} \emph{J. Audio Eng.
%  Soc}, vol.~65, no. 1/2, pp. 42--55 (2017 Jan.Feb.),
%  \doi{https://doi.org/10.17743/jaes.2016.0063}.
%  
%\end{thebibliography}

%Appendix
\appendix
\section*{APPENDIX}
Filtering an audio signal with an allpass filter does not usually have a major effect on the signal's timbre. The allpass filter does not change the frequency content of the signal, but only introduces a phase shift or delay. Audibility of the phase distortion caused by an allpass filter in a sound reproduction system has been a topic of many studies, see, e.g., \cite{DEK1}, \cite{DEK2}.
\begin{equation}
\phi (\omega ) =  - \omega  + 2\arctan \left( {\frac{{a_1 \sin \omega }}{{1 + a_1 \cos \omega }}} \right)
\end{equation}

In this paper, we investigate audio effects processing using high-order allpass filters that consist of many cascaded low-order allpass filters. These filters have long chirp-like impulse responses. When audio and music signals are processed with such a filter, remarkable changes are obtained that are similar to the spectral delay effect  \cite{DEK3}, \cite{DEK4}.


\begin{nomenclature}[PAMPs]
\subsection*{NOMENCLATURE}
\nomentry{a$_c$}{condensation coefficient condensation coefficient condensation coefficient}


\nomentry{TLR}{Toll-like receptor}

\nomentry{PAMPs}{pathogen-associated molecular patterns condensation coefficient condensation}
\end{nomenclature}

%Biography
 \biography{A1firstname A1lastname}{a.eps}{A1firstname A1lastname is professor of audio signal processing at Helsinki University of Technology (TKK), Espoo, Finland. He received his Master of Science in Technology, Licentiate of Science in Technology, and Doctor of Science in Technology degrees in electrical engineering from TKK in 1992, 1994, and 1995, respectively. His doctoral dissertation dealt with fractional delay filters and physical modeling of musical wind instruments. Since 1990, he has worked mostly at TKK with the exception of a few periods. In 1996 he spent six months as a postdoctoral research fellow at the University of Westminster, London, UK. In 2001-2002 he was professor of signal processing at the Pori School of Technology and Economics, Tampere University of Technology, Pori, Finland. During the academic year 2008-2009 he has been on sabbatical and has spent several months as a visiting scholar at the Center for Computer Research in Music and Acoustics (CCRMA), Stanford University, Stanford, CA. His research interests include musical signal processing, digital filter design, and acoustics of musical instruments. Prof. V\"alim\"aki is a senior member of the IEEE Signal Processing Society and is a member of the AES, the Acoustical Society of Finland, and the Finnish Musicological Society. He was the chairman of the 11th International Conference on Digital Audio Effects (DAFx-08), which was held in Espoo, Finland, in 2008.}
 \biography{A2firstname A2lastname}{b.eps}{A2firstname A2lastname is a consulting professor at the Center for Computer Research in Music and Acoustics (CCRMA) in the Music Department at Stanford University where his research interests include audio and music applications of signal and array processing, parameter estimation, and acoustics. From 1999 to 2007, Abel was a co-founder and chief technology officer of the Grammy Award-winning Universal Audio, Inc. He was a researcher at NASA/Ames Research Center, exploring topics in room acoustics and spatial hearing on a grant through the San Jose State University Foundation. Abel was also chief scientist of Crystal River Engineering, Inc., where he developed their positional audio technology, and a lecturer in the Department of Electrical Engineering at Yale University. As an industry consultant, Abel has worked with Apple, FDNY, LSI Logic, NRL, SAIC and Sennheiser, on projects in professional audio, GPS, medical imaging, passive sonar and fire department resource allocation. He holds Ph.D. and M.S. degrees from Stanford University, and an S.B. from MIT, all in electrical engineering. Abel is a Fellow of the Audio Engineering Society.}
\end{document}
