%
% ALTERNATE STORY
%
% 1. ICU benefits from BIG DATA
% 2. MIMIC is the only open access but while very cool it has limitation that a CDM would help  to
% 3. OMOP as been described the best CDM. Recently FHIR is Emmerging.
% 4. Objective threefold are: 1. transform, 2. contribute 3. analysis

% ICU: a paradox 
Intensive Care Units (ICUs) are particularly sensitive units  where demand of
care is rising\cite{angus2000} and mortality is up to 30\% which is a major
health care problem \cite{icu-mortality}. Studies have shown that intensivists
use a limited number of clinical information concepts to guide the
decision\cite{icu-evidence} and that the medical practices are sparse and
variable. Knowing that ICU patients' health record are highly detailed
including connected devices this is a paradox.
% Medical database hopes 
The increasing adoption of Electronic Health Records (EHR) systems worldwide
makes it possible to capture large amounts of clinical data
\cite{bigdata-promise} and big data mining has the potential to play an
important role in clinical medicine \cite{bigdata-mining}. Indeed on the basis
of broad patient medical informations expectations are to improve clinical
outcomes and practices, allow personalized medicine and guide decision thought
early warning systems, and also enrol large and multicentric cohort easily
while minimizing costs.

% Existing ICU databses
By now several commercial or noncommercial, opensource or nonopensource ICU
databases have been developed.
\emph{CUB-REA} is a semi-automatic collection of 25 years and over 300k ICU
patients stays from 30 distinct ICUs in Paris region and covers administrative
data only \cite{cubrea-descr,cubrea-website}. Yet it is subject of 15
international publication and access is under conditions.
\emph{OutcomeREA} is a manual collection of 20 years over 20k UCU patients
stays with medium granularity data from 20 distinct ICUs in France. Yet it is
subject of 50 international publications and access is under conditions.
\emph{eICU} \cite{eicu-website} is an automatic collection of 2 years of 1.5M
ICU patients stays from 500 ICU in the USA and covers high granularity
data \cite{mimic-i2b2}. It is a commercial database.
\emph{MIMIC} (Medical Information Mart for Intensive Care) is a
semi-automatic collection of 10 years and over 60k ICU patients stays with very
high granularity (including EKG) from 1 ICU in Boston. Yet it is subject of 300
international publication. It is a freely-available database.
All those databases have their own dedicated model. Their structural model
are all based on relational database but all do have tables and columns with
different meaning and granularity. As an example MIMIC do have two inputevents
tables reflecting its source center changed its EHR. Also their conceptual
model are mostly different. For example MIMIC do have ICD9 for condition
terminology, while CUB-REA do have both CIM9 and CIM10. 

% Common Data Models
Some studies have shown that using a common data model (CDM) for databases
allows multicentric research, allow mining rare disease  and
catalyse research in general by allowing sharing practices, code, and
tools \cite{cdm-review}. On the other hand some studies have shown that results are
not replicable from one to another database \cite{omop-replicability} and that
keeping the local conceptual model \cite{fhir-deep} and structure
\cite{imi-protect} of database for research leads to better outcomes. Last but
not least, studies have show that CDM can lead to different results
\cite{cdm-comparison}.
% Existing CDM
A dozen of CDM have emerged but we limited the candidate data models to those
designed and used for clinical researches, and those freely available in the
public domains without restrictions.
\emph{OMOP} (Observational Medical Outcomes Partnership Common Data Model) is
a CDM designed for multicentric Drug adverse Event and now enlarges to medical,
clinical and also genomic use cases. OMOP provides both structural (as as set
of relational tables) and conceptual (as a set of standard terminologies) such
SNOMED for diagnoses, RxNORM for drug ingredients and LOINC for laboratory
results. While OMOP has proven its fiability \cite{omop-eval} the fact that
concept mapping process is known to have impact on results
\cite{omop-concept-impact} and that applying the same protocol on different
data sources leads to different results \cite{omop-replicability} reveals the
importance of keeping the local codes to allow local analysis. Several example
of transforming databases into OMOP have been published
\cite{omop-german,omop-nashville} and yet OMOP stores 682 milion patients
records from all over the world\cite{omop-bigboy}. OMOP had 5 versions, and
prones its strong backward compatibility.
\emph{i2b2/SHRINE} is a medical cohort discovery tool used in more than 200
hospitals over the world. SHRINE is one of the attempt to federate multiple
instances of i2b2. The i2b2 star schema has proven its high flexibility thanks
to the modular design of the fact tables allowing storing numerics, characters
or concepts. Its single terminology model is a path based hierarchical table
does not allow to modelise graph ontology (such snomed). While i2b2 is highly
efficient for cohort discovery, it's model wasn't designed for ad-hoc analysis.
A bidirectional terminology mapping \cite{shrine-design} between local concepts
and the SHRINE core set of standard concepts is a prerequirement to participate
to a SHRINE network.
\emph{HL7-FHIR} (Fast Healthcare Interoperability Resources) is a medical
data exchange API specification. FHIR provides a structural CDM that can be
materialized as JSON, XML or RDF format. FHIR is flexible and does not specify
a standard conceptual model so that each hospital can add extension to
implement specific data or share within it's local terminology making each FHIR
implementation sensibly divergent. While some research show it as a promising
CDM for ad-hoc analysis \cite{fhir-google} or cohort discovery
\cite{fhir-paris}, its graph nature adds a layer of transformation making usage
complicated for data-scientists as well as difficult to create standardized
analysis. Finally the model envolves and does not make the asumption of
backward compatibilities along the versions.
%- PCORnet, the National Patient-Centered Clinical Research Network (http://pcornet.org/pcornet-common-data-model/)
%        - have been created to monitore the safety of FDA-regulated medical products.
%	- PCORnet Common Data Model (CDM) integrate multiple data from different sources and leverages standard terminologies and coding systems for healthcare (including ICD, SNOMED, CPT, HCPSC, and LOINC) to enable interoperability with and responsiveness to evolving data standards.
%	- The first version of the CDM was released in 2014, and there have been 3 major releases and one minor update since then (last release CDM v4.1: Released May 18, 2018 )

% CHOICE (OMOP & MIMIC)
% choice of mimic
To demonstrate the feasibility and the opportunity to use a CDM for ICU
databases, we choose MIMIC since it is freely accessible and then the result of
this work will be accessible for subsequent improvement, analysis, and
demonstration. Secondly, MIMIC does have the most broad and high granularity
database leading to evaluate the ability of the CDM to ingest such complex
dataset. More importantly, MIMIC stated they would think of a
CDM\cite{mimic-nature}, that would solve some structure and conceptual
weaknesses of MIMIC mentionned above and this work should help MIMIC in it's
goal to build a international ICU database \cite{mimic-i2b2}.
% choice of omop
Among the other CDM, OMOP looks like the best fit as it allows both
multicentric standardised analysis as well as monocentric specific modeling and
analysis. 
Compared to PCORnet \cite{omop-vs-pcornet} OMOP performs better in the
evaluation database criteria compared with the other models (and PCORnet in
particularly) : completeness, integrity, flexibility, simplicity of
integration, and implementability, accommodate the broadest coverage of
standard terminologies, provides more systematic analysis with analytic library
and visualizing tools, provides easier SQL models.
Compared to i2b2/SHRINE, the unilateral mapping methodology of OMOP is more
effective than the bidirectional \cite{shrine-design} SHRINE mapping. Hence
OMOP proposes a broader set of standardised concepts. In terms of structural
CDM OMOP is highly rigourous in how data shall be loaded in a particular table
when i2b2 is highly flexible with a one general table that solution every data
domains. This rigourous approach is necessary for standardisation. Previous
work have loaded i2b2 with MIMICiii \cite{mimic-i2b2} - however, the concept
mapping step have limited the results since i2b2 design do not store the local
ontology or information where OMOP design allows to keep the concept mapping
unfinished. OMOP has this advantage to keep the terminology mapping step not
mandatory by keeping the local codes in a usable format.  As compared to FHIR,
OMOP performs better as a conceptual CDM as FHIR does not specify which
terminology to implement. In terms of structural CDM OMOP relational model can
be materialized into csv format and stored in any relational database when FHIR
materialised as json needs some processing and more skills to be exploited.
OMOP shares the advantages of all above models. 
Previous preliminary work have been made on translating MIMIC into OMOP
\cite{mimic-omop-previous} and still work remains to be affined and upgraded to
be furthermore evaluated.

% Evaluation of MIMIC-OMOP
In order to evaluate the MIMIC-OMOP transformation we propose to answer to
remaining question such the difficulty of transforming/maintaining an OMOP
dataset from an existing one, how well the initial dataset is integrated and
how much data is lost in the process, how clear and simple the model is to be
queried simply and efficiently by scientists, how well design it is to be
enriched by collaborative work, and finally in what extend OMOP can integrate
and makes feedbacks to intensivists in a realtime context.  This work is then
evaluated through 3 axes: Transformation, Contribution and Analytics.
% contribution
The \emph{first} major contribution of this study is to evaluate OMOP in a real
life and well known freely accessible database. The \emph{second} major
contribution is to provide a freely accessible dataset in the OMOP format that
might be useful for researchers. The \emph{third} major contribution is to
provide the OMOP community some useful transformations dedicated to ICU and
that can be reused in any OMOP dataset.
