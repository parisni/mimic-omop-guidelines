%
% 1. ICU and BIGDATA
% 2. MIMIC DB & its limitations
% 3. CDM can overcome this limitations (and are better than FHIR or I2B2 thanks to standard concepts)
% 4. OMOP is better
% 5. Evaluation method and importance of the work
%    Objective threefold are: 1. transform, 2. contribute 3. analysis
%
% ICU and BIGDATA
Intensive care units (ICUs) are care units where the demand for care 
increases\cite{angus2000} while mortality reaches up to 30\%, 
which is a major health problem\cite{icu-mortality}. 
Studies have shown that intensivists use a limited level of evidence 
to guide decision making\cite{icu-evidence} 
and that medical practices are sparse and variable.
Knowing that the ICU patient health record is very detailed and that there is 
a high density environment for data production is a paradox. 
The increasing adoption of electronic health record (EHR) systems 
around the world is capturing large amounts of clinical data\cite{bigdata-promise} 
and data mining has the potential to play an important role 
in clinical medicine\cite{bigdata-mining}. 
Indeed, based on important medical informations, expectations are 
to improve clinical outcomes and practices, 
enable personalized medicine and guide early warning systems, 
and also easily enroll a large, multi-center cohort while minimizing costs.

% MIMIC DB
% - local terminologies
% - one center
% - two successive critical care information systems (CCI)
% - a model that reflect the EHR
\emph{MIMIC} (Medical Information Mart for Intensive Care) 
is a 10 year semi-automatic dataset of over 60,000 intensive care stays 
with very high granularity (including EKG) 
from two successive intensive care information systems (CCI)
 at the Beth Israel Deaconess Medical Center in Boston. 
It is the first ICU database available free of charge and has been the subject 
of more than 300 international publications. 
However, its monocentric nature makes it difficult to generalize findings to other ICUs. 
The MIMIC relational data model reflects the original CCI, 
as evidenced by the two separate \textit{inputevent\_mv} and \textit{ouputevent\_cv} \cite{mimic-nature}
or the two separate terminologies for physiological data. 
This leads analysts (datascientists, statisticians, etc.) 
to reconcile this heterogeneity when pre-processing each study.

% Why a Common Data Models
Some studies have shown that using a common data model (CDM) by generalizing 
the structural (data model) and conceptual (terminological model) design database 
allows for multicentre research, exploitation of rare diseases 
and catalyzes research by sharing practices, source code and tools \cite{cdm-review,data-enclave}. 
As Kahn and Al said \cite(kahn_data_2012),  "databases modelling is the process 
of determining how data are to be stored in a database". 
It specifies data types, constraints, relationship and metadata definitions 
and provides a standardized way to represent resources/data and their relationships. 
However, some studies have shown that the results are not fully reproducible 
from one CDM to another \cite{cdm-comparison} or from one centre to another \cite{omop-replicability}. 
The lighter approaches argue that maintaining the local conceptual model \cite{fhir-deep}
or the original conceptual and structural model \cite{imi-protect} of the research database 
leads to better results. On the one hand, keeping MIMIC on its specific form 
will not solve the limitation for multicentric research and, on the other hand, 
a fully standardized form would introduce other disadvantages. 
The ideal solution is probably in between to allow 
local or standardized analysis depending on the research question.

% OMOP
\textit{OMOP} (Observational Medical Outcomes Partnership Common Data Model) 
is a CDM originally designed for multi-centre drug-related adverse events 
and now extends to medical, clinical and genomic cases. 
OMOP provides structural and conceptual models such as SNOMED for diagnostics, 
RxNORM for drugs and LOINC for laboratory results. Several examples of database 
transformation to OMOP have been published \cite{omop-german,omop-nashville} 
and OMOP stores 682 million patient records from around the world \cite{omop-bigboy}. 
Each clinical area is stored in different dedicated tables. 
The OMOP conceptual model is based on a closure table pattern \cite{closure-table} 
capable of ingesting any simple, hierarchical and also graph terminologies 
such as SNOMED-CT. In addition to local terminologies, OMOP specifies 
and maintains a set of standard terminologies to be mapped unidirectionally 
(local to standard) by implementers.
Although OMOP has proven its reliability \cite{omop-eval}, 
the concept mapping process is known to have an impact on results \cite{omop-concept-impact} 
and the application of the same protocol on different data sources 
leads to different results \cite{omop-replicability}. 
This shows the importance of keeping local codes so that local analysis is always possible.
Previous preliminary work has been done on the translation of MIMIC into OMOP \cite{mimic-omop-previous}. This work remains to be refined and updated to be evaluated.

% CDM Comparison
In a recent CDM comparative study \cite{cdm-review,omop-vs-pcornet}
OMOP obtained better results in the criteria of the evaluation database 
compared to the other models: completeness, integrity, flexibility, simplicity 
of integration and implementability, adapting to the wider coverage 
of standard terminologies, providing a more systematic analysis 
with an analytical library and visualization tools, providing SQL models easier to use.
That is why OMOP offers a broader set of standardised concepts.
In terms of structural CDM, OMOP is very rigorous in how data should be loaded 
into a particular table when i2b2 for example is very flexible 
with a general table that solves all data domains. This rigorous approach 
is necessary for standardization. 
Previous work has loaded i2b2 with MIMIC-III \cite{mimic-i2b2} - 
however, the concept mapping step has limited the results since i2b2 design 
does not store local ontologies or informations where OMOP design keeps 
concept mapping unfinished. OMOP has the advantage of not making the terminology 
mapping step mandatory by keeping local codes in a usable format. 
Compared to the FHIR, OMOP performs better as a conceptual CDM because 
the FHIR does not specify the terminology to be used. In terms of structural CDM, 
the OMOP relational model can be materialized in csv format
 and stored in any relational database when FHIR uses json files and 
needs some processing and more skills to exploit.
We believe OMOP shares the advantages of all the above models.

% Evaluation of MIMIC-OMOP
In order to evaluate the transformation of MIMIC into OMOP, 
we propose to answer the following questions, 
such as the difficulty of transforming/maintaining an OMOP dataset 
from an local database, how the initial data is integrated 
and how much data is lost in the process, how the model should be queried simply 
and efficiently by analysts, how the design should be enriched by collaborative work, 
and finally to what extent OMOP can integrate and feed back to intensivists 
in a real-time context. This work is then evaluated according to 3 axes: 
Transformations, Contributions and Analyses. 
% contribution
The \emph{first} major contribution of this study is to evaluate OMOP in a freely 
accessible and well known database. The \emph{second} major contribution is to provide 
a freely accessible dataset in OMOP format that could be useful to researchers. 
The \emph{third} major contribution is to provide the OMOP community with useful 
transformations dedicated to intensive care that can be reused on any OMOP data set.
