\subsection{Transformation}
% effectiveness of the transformation
The ETL (including unit tests, and generation of the ready to load archive) on
the 100 patient subset takes five minutes and allows fast cycles of
developments. On the full MIMIC dataset the ETL last 3 hours. 



  For the prescritions MIMIC-III table 75% (a verifier) of drugs had a gsn code. The conept_relationship table provide mapping between gsn and RxNorm classsifications. To improve the mapping we then proceeded to a manual mapping

\subsection{Analytics}

\subsection{Contribution}

The \emph{denormalized derived} tables appears to improves the computations
costs and the verbosity of the SQL queries. In addition the resulting tables
are much more human readily with the label of the concepts directly in the fact
table. Hence a bit of denormalization improves greatly the experience of the
data scientist by adding some redonduncy in the data while not breaking
existing SQL queries. The drawback is it would make it more tricky to update
and keep the dataset consistent - which is non applicable because OMOP is
generally a static dataset.
The \emph{microbiology derived} table simplifies the experience for
data-scientists.
