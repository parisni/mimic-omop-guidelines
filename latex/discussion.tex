% ELT
The computation time of the ETL on the PGSQL instance on a modest personal
computer is compatible with a community work where collaborator can clone the
source code and setup a development instance to reproduce or improve the work.
The choice of the ELT based mostly in SQL code lets end users with SQL
background only to review and enhance the work. As a result, the targeted
community is as broad as possible and we expect translational profiles to get
involved.

% SQL: a choice for future
The datathon has shown that distributed platforms with commodity hardware
provides SQL tools allowing OLAP analysis with great performances that overcome
OLTP RDBMS weaknesses. Hence it takes advantage of SQL language analytics
features such grouping, windowing, joining and mathematic functions that often
lack in NOSQL databases.

% about derived tables
It is important that OMOP keeps a level of normalisation in order to simplify
the ETL and make it consistent. However once done, it is judicious to give
access to data-scientist to more denormalized tables and more specialized
tables. Multiple concerns exists about OMOP performances and optimization.
However there will never be a perfect multi-use case table, and this is the
reponsibility of the data scientist to build his own tables, simplified,
specialized for his research and answer efficiently and clearly his needs.

% About derived data
Derived data integrates quite well in OMOP. We made use of note\_nlp to store
information derived from notes, measurement to store numerical information and
cohort\_attributes to store scores. However it is still unclear if derived data
should be stored per domain or if it should be stored in dedicated derived
tables. We found out that there is a lack of tables to track provenance and
description of such data.

% missing in the model
%% data quality
An other missing aspect is some quality tables to assess and measure the
quality of data. MIMIC had some column to keep track of corrupted information.
It would be of interest to be able to keep the messy data and allow research on
data cleaning and data quality and avoid removing information.

%% mecanisms for real time ingestion of data, such version control
Last but not least, as stated in the introduction a good CDM for ICU would
allow near realtime early warning systems and model inference on fresh data.
OMOP is clearly designed to provide static dataset and does not have mecanisms
for realtime ingestion, and data version control - it is not a datawarehouse.
Making analysis on statics datasets is essential in order to have reproducible
results. However when the algorithm needs to be moved at bed side, there is a
need for a freshness of data, and a way to identify the patient that OMOP won't
easily provide. That being said a solution such FHIR is a great way to
implement realtime inference from EHR data and that's how FHIR and OMOP are
complementary that yet have been investigated \cite{gatech} but needs further
optimizations.

% OMOP community organisation
During this work the OMOP forum was very active. Working groups.  It is a
challenge to manage such large community from all moderator, contributors and
from a user perspective. It appears it is not doable for most of people to get
involved. The forum is full of details and information. It contrasts with the
implementation guide that suffer from not being as well detailed. We think the
OMOP community would greatly benefit from systematic and synthetic
synchronisation between forum, mailing lists,  github and end user
documentation.

% Documentation need
The datathon real life test has revealed the strong need to make the physical
data model including comments on columns and table accessible and we found out
that the open-source tool schema spy was of good help. In addition limit the
misunderstanding by delivering SQL examples and we found out that the git
repository is the best place for documenting and interact with the community.

% On static / early warning system
OMOP has been designed to build static
