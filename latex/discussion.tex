% SQL: a choice for future

% about derived tables
It is important that OMOP keeps a level of normalisation in order to simplify
the ETL and make it consistent. However once done, it is judicious to give
access to data-scientist to more denormalized tables and more specialized
tables. Multiple concerns exists about OMOP performances and optimization.
However there will never be a perfect multi-use case table, and this is the
reponsibility of the data scientist to build his own tables, simplified,
specialized for his research and answer efficiently and clearly his needs.

% About derived data
Derived data integrates quite well in OMOP. We made use of note\_nlp to store
information derived from notes, measurement to store numerical information and
cohort\_attributes to store scores. However it is still unclear if derived data
should be stored per domain or if it should be stored in dedicated derived
tables. We found out that there is a lack of tables to track provenance and
description of such data.

% missing in the model
%% data quality
An other missing aspect is some quality tables to access and measure the
quality of data. MIMIC had some column to keep track of corrupted information.
It would be of interest to be able to keep the messy data and allow research on
data cleaning and data quality and avoid removing information.

%% mecanisms for real time ingestion of data, such version control
Last but not least, as stated in the introduction a good CDM for ICU would
allow near realtime early warning systems and model inference on fresh data.
OMOP clearly does provide static dataset and does not have mecanisms for
realtime ingestion, and data version control - it is not a datawarehouse. That
being said a solution such FHIR is a great way to implement realtime inference
from EHR data and that's how FHIR and OMOP are complementary that yet have been
investigated \cite{gatech}.
