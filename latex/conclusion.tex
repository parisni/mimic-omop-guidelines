% OMOP transformation
The MIMIC to OMOP transformation has needed some efforts that remains
reasonable. It is and will be still a work in progress as the standard concept
mapping is a quite infinite process with constant improvement. Fortunately the
released version is ready for research and already offers the same perimeter of
data as the original MIMIC version and even more with the derived data.

% OMOP analytics efficiency 
As seen OMOP model efficiency has some weaknesses as it looks to put the cursor
more on consistency than on performances. However we have shown that it's easy
to overcome the issues and complete OMOP with a set of design or technology
optimization and dedicated structure that in the end remains standards and
shareable because based on the original model.

% OMOP contributions
As compared to the original MIMIC data model, working on OMOP offers the
opportunity to write standard code and analysis that might benefit from or to
other users internationally. The MIMIC-OMOP database is available online on
physionet as well as the original MIMIC database. All the existing work is
publicly available on github and has been designed to be reviewed, copied or
enriched easily accordingly to OMOP or MIMIC open-source philosophy.

% Further work
Future work on evaluation of the existing concept mapping though practical
research studies on both local and standard coding will be made. In addition we
expect to enhance the OHDSI USAGI concept mapping tool to allow international
concept mapping suggestion.
