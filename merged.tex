\section{ICU datas : a paradox}\label{icu-datas-a-paradox}

\begin{itemize}
\item
  Reusing medical datas has historically been impossible for a large
  population and most of datas were simply wasted due data variability
  and quality challenges
\item
  \begin{description}
  \item[Intensive care unit ICU are faced to a paradox]
  \begin{itemize}
  \tightlist
  \item
    The level of proof to guide most decisions is low, exacerbated with
    real-time bedsite decisions and the medical practices are sparse
    (1).
  \item
    High density environment for data production : prescriptions
    systems, monitoring (waves), ventilators and large number of exams
    done in this units
  \end{itemize}
  \end{description}
\item
  The practice's variability is due to lack of adherence to best
  practices, but the vast majority occurs simply because no evidence has
  been established for the issue in question (2) or because the effects
  of interventions in the ICU are subject to the exceptional complexity
  of patient physiology and the variation beetween unique patient and
  clinical studies
\item
  But the ICU demand of care is rising (3) and the mortality in ICU is
  up to 30 \% which is a major health care problem
\end{itemize}

\section{ICU databases}\label{icu-databases}

\subsection{aim}\label{aim}

The multiple aims were - to create complete and highly detailed patient
record - minimize costs while improving the clinical outcomes of
individuals and populations thanks to observational clinical research
and real time algorithms

\subsection{databases (7)}\label{databases-7}

Several commercial or noncommercial, opensource or nonopensource ICU
databases have been developed

\begin{itemize}
\item
  \begin{description}
  \item[Commercial eICU]
  \begin{itemize}
  \tightlist
  \item
    developped in partenariat with Philips
  \item
    available via PhysioNet
  \item
    over 1.5 million ICU stays
  \item
    and is adding 400,000 patient records per year from over 180
    subscribing hospitals in the country.
  \item
    patients who were admitted to critical care units in 2014 and 2015.
  \item
    Data are heterogenous and high granularity signal as waveform is not
    record
  \end{itemize}
  \end{description}
\item
  \begin{description}
  \item[Non commercial CUB-REA database]
  \begin{itemize}
  \tightlist
  \item
    B. Guidet, P. Aegerte
  \item
    \url{http://www.pifo.uvsq.fr/hebergement/cubrea/cr_index.htm}
  \item
    collected since 25 years around 300k ICU patients stays low
    granularity data from 30 distinct ICUs in Paris region.
  \item
    Data are collected semi automatically annually
  \item
    15 international publications.
  \end{itemize}
  \end{description}
\item
  \begin{description}
  \item[Non commercial OutcomeREA database]
  \begin{itemize}
  \tightlist
  \item
    JF Timsit
  \item
    \url{http://outcomerea.fr/index.php}
  \item
    collected since 20 years around 20k ICU patients stays medium
    granularity data from 20 distinct ICUs in France
  \item
    Data are daily collected manually by senior trained intensivists,
  \item
    This database has been subject of 50 publications.
  \end{itemize}
  \end{description}
\item
  \begin{description}
  \item[Non commercial MIMICIII (Medical Information Mart for Intensive
  Care) : our case study]
  \begin{itemize}
  \item
    R. Marc
  \item
    freely-available database via PhysioNet :
    \url{https://mimic.physionet.org/}
  \item
    Data are collected each 5 years, semi automatically.
  \item
    This database is de-identified and open, and one can exploit the
    data after passing an online exam on clinical ethic.
  \item
    over 300 publications from international researchers independant
    from the MIT
  \item
    health-related data associated with over forty thousand patients who
    stayed in critical care units between 2001 and 2012(4).
  \item
    It includes both administrative data (demographic, ICD9, procedures)
    and clinical data (examination, laboratory results, medication
    administration and notes)
  \item
    \begin{description}
    \item[Three types of data are collected :]
    \begin{itemize}
    \tightlist
    \item
      clinical data from hospital information system,
    \item
      death data from the social security database
    \item
      High granulary data as the waveform of EKG, EEG.
    \end{itemize}
    \end{description}
  \end{itemize}
  \end{description}
\end{itemize}

\subsection{conclusion}\label{conclusion}

The MIMIC-III database is unique in capturing highly granular structured
data. But the conception of this database was time consuming and
unfortunately only 45,000 unique patients' data from a single center
were captured. To produce analyse high number of patient we will have to
merge heterogenous databases.

\section{Data merging}\label{data-merging}

\subsection{aims}\label{aims}

Use of EHRs has been increasing world-wide, but most EHRs are different
in their structure and not interchangeable.

\begin{itemize}
\tightlist
\item
  more data : may provide better outcomes
\item
  interoperability may provide easy international research and improve
  reproductibily of it
\item
  decrease costs and investment in developing algorithms and help to
  performs transferable analyses
\end{itemize}

\subsection{challenges}\label{challenges}

\begin{itemize}
\tightlist
\item
  but we know that simple merging of databases give poor quality level
  because of the heterogeneity of datas (9)
\item
  but sharing data creates legal/juridic problems
\item
  but merge may loss datas
\end{itemize}

\subsection{databases modelling and datas
exchanges}\label{databases-modelling-and-datas-exchanges}

Common data model (CDM) provides standardized definition of represent
resources and their relationships. Many has been developped, certains
are open-source: - OMOP model : Observational Medical Outcomes
Partnership Common Data Model (OMOP-CDM) - incorpore validated standard
classification (8) : SNOMED for diagnoses, RxNORM for drug ingredients
and LOINC for laboratory results\ldots{} - provide tables for mapping
beetween international classification (ex: ICD9 and SNOMED) - provides
more systematic analysis with analytic library from OMOP community :
ACHILLES

\begin{quote}
\begin{itemize}
\tightlist
\item
  In this model all the data stay locally at the participant site, the
  primary analyses are carried locally (5)
\item
  This model has been already adopted by more than 682 million patient
  records with databases from all over the world(9)
\item
  Several examples of transforming source databases to CDM already
  exists (10-11)
\end{itemize}
\end{quote}

\begin{itemize}
\item
  \begin{description}
  \item[I2B2 :]
  \begin{itemize}
  \tightlist
  \item
    good interface for cohort selection
  \item
    i2b2 has been described as being used by more than 200 hospitals6
    over the world
  \item
    The central table is called observation\_fact table.
  \item
    Compare to OMOP-CDM the hierarchies are organise with a `concept
    path' column. Two concepts are linked by a single relationship
  \end{itemize}
  \end{description}
\item
  \begin{description}
  \item[PCORnet, the National Patient-Centered Clinical Research Network
  {[}TODO APA{]}]
  \begin{itemize}
  \tightlist
  \item
    PCORnet Common Data Model (CDM) hoping to integrate multiple data
    from different sources and leverages standard terminologies and
    coding systems for healthcare (including ICD, SNOMED, CPT, HCPSC,
    and LOINC) to enable interoperability with and responsiveness to
    evolving data standards.
  \item
    The first version of the CDM was released in 2014
  \item
    Compare to OMOP CDM, PCORNET is less effective for use with a
    longitudinal community registry (6)
  \end{itemize}
  \end{description}
\item
  \begin{description}
  \item[FHIR, Fast Healthcare Interoperability Resources]
  \begin{itemize}
  \tightlist
  \item
    is a standard for exchanging healthcare information electronically
    (\url{https://www.hl7.org/fhir/overview.html})
  \item
    Some papers have showed that collaboration between FHIR may provide
    both applicative software and analytic research and showed great
    promise(5, 13) nico
  \end{itemize}
  \end{description}
\item
  MIMIC !
\end{itemize}

OMOP choice Justification: {[}TODO APA{]} Terminology standardized,
analytics tools tool available, SQL Model (Justifier VS NO-SQL).

\section{Our study}\label{our-study}

The aim of MIT with MIMIC-III is to provide open datas, more
collaborative and reproductitible studies with shared codes. In this
purpose the transformation from MIMICIII to MIMICIII-OMOP with
standardized mapping concept is important and was hightly supported by
the MIT. (4)

In this article we provide a example of Extract Transform Load (ELT)
implementation of electronic health records (EHR) in intensive care unit
by transforming the all MIMIC-III database (expected high frequency
datas) to OMOP CDM version 5.3 (last version in date). We'll expose our
methodology and we'll discuss about modification we want to propose to
the omop community. We'll also discuss about potential loss of
information links to this ETL.

3 axes of evaluation : ETL, ANALITICS, Contribution.

1. Vincent JL. Is the current management of severe sepsis and septic
shock really evidence based? PLoS Med 2006; 3:e346 2. Vincent JL, Singer
M. Critical care: advances and future perspectives. Lancet 2010;
376:1354--1361 3. Angus DC, Kelley MA, Schmitz RJ, White A, Popovich J
Jr; Committee on Manpower for Pulmonary and Critical Care Societies
(COMPACCS). Caring for the critically ill patient. Current and projected
workforce equirements for care of the critically ill and patients with
pulmonary disease: can we meet the requirements of an aging population?
JAMA 2000;284:2762--2770 4. A.E.W. Johnson, Tom J. Pollard and Al.
MIMIC-III, a freely accessible critical care database. Scientific Data.
2016-5-24 5. M. Choi and Al. OHDSI on FHIR Platform Development with
OMOP CDM mapping to FHIR Resources,Georgia Tech Research Institute,
poster 6. M.Garza. Evaluating common data models for use with a
longitudinal community registry. Journal of Biomedical Informatics 2016.
333--341 7. Jeff Marshall, Abdullah Chahin and Barret Rush. Chapter 2
Review of Clinical Databases - Springer 8. JM Overhage and Al.
Validation of a common data model for active safety surveillance
research. J Am Med Inform Assoc. J Am Med Inform Assoc 2012;19: 54-60 9.
G. Hripcsak and Al. Observational Health Data Sciences and Informatics
(OHDSI): Opportunities for Observational Researchers.Stud Health Technol
Inform. 2015 ; 216: 574--578 10. F. FitzHenry and Al. Creating a Common
Data Model for Comparative Effectiveness with the Observational Medical
Outcomes Partnership. Appl Clin Inform 2015; 6: 536--547 11. S. Bayzid
and Al. Conversion of MIMIC to OHDSI CDM. National Center for Biomedical
Communications, Bethesda, Maryland 12. T. Gruber. Toward principles for
the design of ontologies used for knowledge sharing?, International
journal of human-computer studies, 1995 13. Nicolas Paris and Al. i2b2
implemented over SMART-on-FHIR data source =\textgreater{} intro
\#\#\#\#\#\#\#\#\#\#\#\#\#\#\#\#\#\#\#\#\#\#

\begin{itemize}
\item
  \begin{description}
  \item[several other open-source databases]
  \begin{itemize}
  \tightlist
  \item
    eICU (3), freely-available comprising deidentified with more than
    hundreds of thousands of patients. Data are available to researchers
    via PhysioNet, similar to the MIMIC database
  \item
    OUTCOMEREA (\url{http://outcomerea.fr/index.php})
  \item
    CUBREA
    (\url{http://www.pifo.uvsq.fr/hebergement/cubrea/cr_index.htm}),
    with many ICU from APHP with \textgreater{} 2000000 icu stays
  \end{itemize}
  \end{description}
\item
  presentation of mimicIII : our case study MIMIC-III (Medical
  Information Mart for Intensive Care) is freely-available database
  comprising deidentified health-related data associated with over forty
  thousand patients who stayed in critical care units between 2001 and
  2012(1). It includes both administrative data (demographic, ICD9,
  procedures) and clinical data (examination, laboratory results,
  medication administration and notes) Three types of data are collected
  : clinical data from hospital information system, death data from the
  social security database and the high granulary data as the waveform
  of EKG, EEG. In this article we won't speak about high frequency
  datas.

  The aim of MIT with MIMIC-III is to provide open datas, more
  collaborative and reproductitible studies with shared codes. MIMIC is
  a large used database with x number of publications. In this purpose
  the transformation from MIMICIII to MIMICIII-OMOP with standardized
  mapping concept is important. The mimic documentation is a available
  online physionet.org/about/mimic/. A public github was created :
  \url{https://github.com/MIT-LCP/mimic-code} with many contributers
  around the world.
\end{itemize}

\section{ETL mapping specifications}\label{etl-mapping-specifications}

\begin{itemize}
\tightlist
\item
  The key table for omop is the concept table. The standard vocabulary
  of OMOP is mainly based on the Systematized Nomenclature of Medicine
  Clinical Terms (SNOMED-CT)
\item
  A mapping between many classification and the standard omop ones
  (ICD-9 and snomed-CT for examples) is already provides with
  concept\_relationship.
\item
  Local code for mimiciii such as admission diagnoses, demographic
  status, drugs, signs and symptoms were manually mapped to OMOP
  standard models by several participants. For example local drug codes
  were mapped to the OMOP standardized vocabularies, which use RxNorm.
  This work was followed and check by a physician. All laboratory exams,
  exit diagnoses and procedures were already mapped to standard
  classication. All the csv files used for the mapping are available on
  github: evaluation + comments fields. =\textgreater{} solution that
  can scale for medical users without database background. {[}TODO
  APA{]}
\end{itemize}

- fuzzy match algorithm for mapping suggestion semi-automatic. {[}TODO:
NPA{]} The manual terminology mapping has been catalized by using a
naïve but flexible approach. Many mapping tools exist on the area RELMA
provided by LOINC, USAGI provided by OHDSI. Most of those tools are
based on linguistic mapping {[}cite{]}, and the approach have been shown
to be the most effective{[}cite{]}. Following our prime idea to build
low dependency tools, we managed to build a light semi-automatic tool
based on postgresql full-text feature. The concept table labels have
been indexed, and a similarity can be constructed by a simple sql query.
We kept the 10 most similar concepts, and this have been shown to be a
quick way to map concepts, after having choosen the best domain.

\section{methodology of ETL}\label{methodology-of-etl}

All the process is available freely on the github website.

\subsection{Preprocessing and modification of
mimic}\label{preprocessing-and-modification-of-mimic}

\begin{itemize}
\tightlist
\item
  We added emergency stays as as a normal locations for patients
  throughout their hospital stay.
\item
  Icustays mimic table was deleted as it is a derived table from
  transfers table (2) and we decided to assigne a new new visit\_detail
  pour each stay in ICU (based of the transfers table) whereas mimic
  prefered to assgned new icustay stay if a new admissions occurs
  \textgreater{} 24h after the end of the previous stay
\item
  We decided to put unique number for each row of mimic database called
  mimic\_id. We think this is very helpful for ETLers
\end{itemize}

\subsection{Technical specifications}\label{technical-specifications}

\begin{itemize}
\tightlist
\item
  To provide standard and reproductilable precess all the ETL used SQL
  script.
\item
  subset of 100 patients,
\item
  unit testing during the all process of extraction and SQL script
  production
\item
  we tried not to infer results. For examens whereas it's logical to put
  a specimen for many labevents results (as one sample of blood may be
  used to multilple exams) we decided to create as many specimen row as
  laboratory exams because the information is not present. It was the
  same when date information were not provide ( start/end\_datetime for
  drug\_exposure)
\item
  concept-driven methodology : as the omop model did we adopt a
  ``concept-driven methodology'', domain of each local concept drive the
  concept to the right table.
\end{itemize}

-mapping : omop model provide mapping between international
classifications, such as gsn and rxnorm, icd9 and snomed thanks to
concept\_relationship table. We used it. Because drugs, chartevents,
specimen, gender, ethnicity are not linked to concept in mimic, concepts
has been mapped by medical doctors. We will evaluate this mapping in the
result section.

\begin{itemize}
\item
  \begin{description}
  \item[fact\_relationship : for drug solution, microbiology /
  antibiograms, visit\_detail and caresite]
  \begin{itemize}
  \tightlist
  \item
    for example : microorganism are links to their antibiogram thanks to
    fact\_relationship \textless{}!-- fournir un exemple de SQL pour ca
    avec un resultat\textgreater{}
  \end{itemize}
  \end{description}
\end{itemize}

\subsection{modification of OMOP
model}\label{modification-of-omop-model}

\begin{itemize}
\item
  the less possible
\item
  keep in mind the model of omop as a conceptual model
\item
  constant dialogue with omop community (omop github, ETL community
  (bresilian))
\item
  \begin{description}
  \item[modifications of OMOP model (few columns)]
  \begin{itemize}
  \item
    \begin{description}
    \item[structural (columns type, columns name, new columns)]
    \begin{itemize}
    \item
      \begin{description}
      \item[visit\_detail/visit\_occurrence : add
      admitting\_source\_value, admitting\_source\_concept\_id,
      admitting\_concept\_id, discharge\_to\_source\_value,
      discharge\_to\_source\_concept\_id, discharge\_to\_concept\_id]
      \begin{itemize}
      \tightlist
      \item
        drug\_strength, drug\_exposure, drug\_era, dose\_era: temporal
        columns.
      \item
        note\_nlp
      \end{itemize}
      \end{description}
    \end{itemize}
    \end{description}
  \item
    \begin{description}
    \item[conceptual (new concepts specific to ICU or general)]
    \begin{itemize}
    \tightlist
    \item
      measurement\_type\_concept\_id
    \end{itemize}

    - the actual visit\_detail doesn't introduce pertinent information
    and duplicate informations from visit\_occurrence table. For
    admitting\_from\_concept\_id and discharge\_to\_concept\_id, we
    extended the dictionary in order to track bed transfers and ward
    transfers. For visit\_type\_concept\_id we assigned a new concept
    for any level of granularity necessary for your use case (ward,
    bed\ldots{}) \textless{}!-- Fournir un example de
    visit\_detail--\textgreater{}
    \end{description}
  \end{itemize}
  \end{description}
\item
  \begin{description}
  \item[modification of MIMIC]
  \begin{itemize}
  \tightlist
  \item
    visit\_detail : admitting\_source\_value,
    admitting\_source\_concept\_id, admitting\_concept\_id,
    discharge\_to\_source\_value, discharge\_to\_source\_concept\_id,
    discharge\_to\_concept\_id provide redondant information from
    visit\_occurrence. We did't populate it.
  \item
    observation\_period provide duplicate information: we fill this
    table to respect the omop model and tools
  \item
    operators have been extracted to fill operator\_concept\_id
  \item
    units of measures have been extracted to fill unit\_concept\_id
  \end{itemize}
  \end{description}
\end{itemize}

\begin{enumerate}
\def\labelenumi{\arabic{enumi}.}
\tightlist
\item
  A.E.W. Johnson, Tom J. Pollard and Al. MIMIC-III, a freely accessible
  critical care database. Scientific Data. 2016-5-24
\item
  \url{https://mimic.physionet.org/mimictables/icustays/}
\end{enumerate}

\subsection{Additional structural
contributions}\label{additional-structural-contributions}

\begin{itemize}
\item
  \begin{description}
  \tightlist
  \item[era/analytics material views]
  - adding concept\_names everywhere for readibility -{[}TODO APA{]}
  microbiology era table

  \begin{itemize}
  \tightlist
  \item
    design specific table for: labs, microbiology, to split measurement
    table into smaller pieces.
  \end{itemize}
  \end{description}
\end{itemize}

- {[}TODO NPA{]} derived data pipelines: methods based on uima. The
note\_nlp table allows to store NLP results derived from plain text
notes. In order to evaluate this table we provided 3 pipelines based on
apache UIMA {[}cite{]} The first pipeline ``section extractor'' splits
the notes into sections in order to help analysts to choose or avoid
some sections from their analysis. The sections patterns (such ``Illness
History'') have been automatically extracted from texts from regular
expressions, automatically filtered by keeping only one with frequency
higher than 1 percent and manually filtered to exclude false positives
with a total of 1200 sections. The resulting sections patterns candidate
have been then manually regrouped into similar 400 groups. The second
pipeline ``tokenizer pipeline'' pre-splits sections into sentences and
tokens. This allows analysts to simply get the tokens by splitting them
by space character. The third pipeline ``n2c2 mi'' extracts information
about myocardial infarction. It states if is negated, from a family
member, and tries to date that fact. The overall performance of the
method has resulted into a 0.97 recall and 0.60 precision measured
during the n2c2 challenge {[}cite{]} The extracted sections have not
been mapped to the any standard terminology such LOINC CDO. The reason
is the CDO LOINC has decided to stop to maintain and to remove it's
sections from its standard arguing it is too difficult to maintain, and
this sections are not widely used
{[}\url{https://loinc.org/news/loinc-version-2-63-and-relma-version-6-22-are-now-available/}{]}.
1. ETL

\begin{enumerate}
\def\labelenumi{\arabic{enumi}.}
\setcounter{enumi}{1}
\tightlist
\item
  ANALYTICS
\end{enumerate}

\begin{itemize}
\tightlist
\item
  consize model, simple
\item
  normalized, but materialized views is a solution.
\item
  standardized code
\end{itemize}

\begin{enumerate}
\def\labelenumi{\arabic{enumi}.}
\setcounter{enumi}{2}
\tightlist
\item
  CONTRIB
\end{enumerate}

\begin{itemize}
\tightlist
\item
  sopha from dataforgood ?
\end{itemize}

NLP: The

\subsection{summary table of note and section
mapping}\label{summary-table-of-note-and-section-mapping}

\begin{description}
\item[with tmp as (select count(1) as count,round(median(c)) as median,
round(avg(c),1) as avg, max(c) as max, note\_source\_value as
mimic\_category, c1.concept\_name as omop\_category from note left join
concept c1 on note\_type\_concept\_id = c1.concept\_id left join (select
note\_id, count(1) as c from note\_nlp group by note\_id) as note\_nlp
using (note\_id) group by note\_source\_value, c1.concept\_name) select
mimic\_category, omop\_category, count as document\_count, median as
section\_median, avg as section\_mean, max as section\_max from tmp
order by 2 asc;]
mimic\_category \textbar{} omop\_category \textbar{} document\_count
\textbar{} section\_median \textbar{} section\_mean \textbar{}
section\_max
\item[-------------------+-------------------+----------------+----------------+--------------+-------------
Case Management \textbar{} Ancillary report \textbar{} 953 \textbar{} 5
\textbar{} 6.3 \textbar{} 16]
Nutrition \textbar{} Ancillary report \textbar{} 9400 \textbar{} 8
\textbar{} 9.6 \textbar{} 23 Pharmacy \textbar{} Ancillary report
\textbar{} 101 \textbar{} 3 \textbar{} 2.3 \textbar{} 3 Rehab Services
\textbar{} Ancillary report \textbar{} 5408 \textbar{} 20 \textbar{}
23.5 \textbar{} 74 Respiratory \textbar{} Ancillary report \textbar{}
31701 \textbar{} 24 \textbar{} 24.1 \textbar{} 35 Social Work \textbar{}
Ancillary report \textbar{} 2661 \textbar{} 2 \textbar{} 7.2 \textbar{}
23 Discharge summary \textbar{} Discharge summary \textbar{} 59652
\textbar{} 29 \textbar{} 28.0 \textbar{} 76 Physician \textbar{}
Inpatient note \textbar{} 141281 \textbar{} 56 \textbar{} 56.3
\textbar{} 98 General \textbar{} Inpatient note \textbar{} 8236
\textbar{} 2 \textbar{} 6.5 \textbar{} 82 Consult \textbar{} Inpatient
note \textbar{} 98 \textbar{} 43 \textbar{} 37.5 \textbar{} 63 Nursing
\textbar{} Nursing report \textbar{} 223182 \textbar{} 1 \textbar{} 3.2
\textbar{} 49 Nursing/other \textbar{} Nursing report \textbar{} 822497
\textbar{} 1 \textbar{} 1.0 \textbar{} 1 ECG \textbar{} Pathology report
\textbar{} 209051 \textbar{} 1 \textbar{} 1.0 \textbar{} 1 Echo
\textbar{} Pathology report \textbar{} 45794 \textbar{} 21 \textbar{}
20.5 \textbar{} 25 Radiology \textbar{} Radiology report \textbar{}
522279 \textbar{} 5 \textbar{} 5.7 \textbar{}
\end{description}

Tokenizer evaluation: The stanford parser have been evaluated in several
studies. The ctakes parser has a specialized Myocardial infaction
evaluation: Last but not least, this pipeline exploits two pipelines
described above. It's evaluation thought a challenge testifies the
approach works and might benefit from improvements. All those NLP
pipelines are interdependent. Improving one step would result in general
improvement. Community work might apply here and subsequent result be
used into cohort discovery or data-science feature extraction by analyst
without prior knowledge in NLP. In order to be able to improve NLP
results, an evaluation framework need to be built up. The NOTE\_NLP
table might be populated with gold standard manually annotated notes
too. While sections, sentences, and token are intermediary results, we
believe that is is important to store them. This has several advantages:
it helps text-miners. This has a severe drawback: the table becomes huge
with potentially billions of rows POS tagging for each token.

\subsection{table populated with their mimic source table
link}\label{table-populated-with-their-mimic-source-table-link}

The OMOP-CDM contains n data tables. We populated m tables. From
MIMICIII we create a standardized model called MIMICIII-OMOP.

\textbar{} Omop tables \textbar{} Source tables\textbar{}
\textbar{}-----------------------\textbar{}--------------\textbar{}
\textbar{} PERSONS \textbar{} patients, admissions \textbar{} \textbar{}
DEATH \textbar{} patients, admissions \textbar{} \textbar{}
VISIT\_OCCURRENCE \textbar{} admissions \textbar{} \textbar{}
VISIT\_DETAIL \textbar{} transfers, service \textbar{} \textbar{}
MEASUREMENT \textbar{} chartevents, labevents, microbiologyevents,
outputevents \textbar{} \textbar{} OBSERVATION \textbar{} admissions,
chartevents, datetimevvents, drgcodes \textbar{} \textbar{}
DRUG\_EXPOSURE \textbar{} prescriptions, inputevents\_cv,
inputevents\_mv\textbar{} \textbar{} PROCEDURE\_OCCURRENCE \textbar{}
cptevents, procedureevents\_mv, procedure\_icd\textbar{} \textbar{}
CONDITION\_OCCURRENCE \textbar{} admissions, diagnosis\_icd \textbar{}
\textbar{} NOTE \textbar{} notevents\textbar{} \textbar{} NOTE\_NLP
\textbar{} noteevents \textbar{} \textbar{} COHORT\_ATTRIBUTE \textbar{}
callout \textbar{} \textbar{} CARE\_SITE \textbar{} trasnfers, service
\textbar{} \textbar{} PROVIDER \textbar{} caregivers \textbar{}
\textbar{} OBSERVATION\_PERIOD \textbar{} patients, admissions
\textbar{} \textbar{} SPECIMEN \textbar{} chartevents, labevents,
microbiologyevents \textbar{}

\begin{itemize}
\tightlist
\item
  observation\_period provide duplicate informations from
  visit\_occurrence : we fill this table to respect the omop model and
  tools
\end{itemize}

\# Quality evaluation

\#\# comparaison MIMICIII / MIMIC OMOP (basic statistics) The table
lists the baseline characterization of the population of MIMICIII-OMOP
compared with MIMICIII.

\textbar{} items \textbar{}OMOP-MIMIC \textbar{} MIMICIII \textbar{}
\textbar{}---------------------------------------\textbar{}-------------------------------\textbar{}----------\textbar{}
\textbar{} Persons (Number) \textbar{} 46.520 \textbar{} 46.520
\textbar{} \textbar{} Admissions (Number) \textbar{} 58.976 \textbar{}
58.976 \textbar{} \textbar{} Icustays (Number) \textbar{} 61.532
\textbar{} 71.576 \textbar{} \textbar{} Age (Mean) \textbar{} 64 ans, 4
months \textbar{} 64 years, 4 monts \textbar{} \textbar{} Gender, Female
(Number, \%) \textbar{} 20.399 (43 \%) \textbar{} 20.399 \textbar{}
\textbar{} Length of stay, hospital (median) \textbar{} 6.59 (Q1-Q3 :
3.84 - 11.88) \textbar{} 6.46 (Q1-Q3 : 3.74 -11.79) \textbar{}
\textbar{} Length of stay, ICU (median) \textbar{} 1.87 (Q1-Q3 : 0.95 -
3.87) \textbar{} 2.09 (Q1-Q3 : 1.10 - 4.48) \textbar{} \textbar{}
Mortality, ICU (Number, \%) \textbar{} 5815 (9\%) \textbar{} 5814 (9\%)
\textbar{} \textbar{} Mortality, hospital (Number, \%) \textbar{} 4559
(6\%) \textbar{} 4511 (7\%) \textbar{} \textbar{} Lab measurement per
admissions (mean) \textbar{} \textbar{} \textbar{}

papier + test

cf extra : basic\_statistics.sql

\#\# loss of data (try to quantify it) - percent of records loaded from
the source database to the CDM - percent of columns - percent of rows as
have done other studies (1)

\begin{itemize}
\tightlist
\item
  Row
\end{itemize}

\textbar{} items \textbar{}rows per persons\textbar{}
\textbar{}-----------------------------------\textbar{}----------------\textbar{}
\textbar{} Nb patients \textbar{} 100 \% \textbar{} \textbar{} Nb
admissions \textbar{} 100 \% \textbar{} \textbar{} Procedures \textbar{}
\% \textbar{} \textbar{} Admissions diagnosis \textbar{} \% \textbar{}
\textbar{} Exit diagnosis \textbar{} \% \textbar{} \textbar{} Laboratory
exams \textbar{} \% \textbar{} \textbar{} Physical exams \textbar{} \%
\textbar{} \textbar{} Drugs \textbar{} \% \textbar{} \textbar{} Notes
\textbar{} \% \textbar{}

remark all the error rows are deleted ( prescriptions, inputevents\_mv,
chartevents, procedureevents\_mv, note)

- Columns \% of sources columns which doesn't fits to CDM storetime!!

\#\# terminology mapping coverage - ICD-9-CM A part of source data for
condition\_occurrence was ICD-9 codes. The OMOP common standard
vocabulary, SNOMED-CT, did not cover all ICD-9-CM codes (95\%) Moreover,
not all ICD-9-CM codes can have one-to-one mapping to SNOMED, some are
one-to-many (28\%)(2) - LOINC - RxNorm

- \% of standard\_concept\_id = 0 (No mapping concept) per table Need
colaborative work

\begin{itemize}
\item
  \begin{description}
  \item[\% of domain\_id not in adequation with table name]
  \begin{itemize}
  \tightlist
  \item
    some are logical because observation domain may be measurement table
    and vice verca
  \end{itemize}
  \end{description}
\item
  we have mapped many source concept to one standard concept is it the
  same meaning? distribution of values sometimes very different
\end{itemize}

\section{ACHILLES evaluation}\label{achilles-evaluation}

ACHILLES is open-source software application developped by OHDSI and
Achilles Heel provided data quality checker Other team used this tool to
practice data quality assess(4). Our result \ldots{} - Quality control -
18h 50k patients: this testifies the model needs structural
optimisations - difficulté pour ajoute fr. - extension achilles how to ?
- comparison with other paper about error/warnings.

\subsection{Community sharing}\label{community-sharing}

We provided many derived values. Community is welcome to improve it -
F/P, corrected Ca / K, BMI - Note\_NLP with section splitting. The
algorythm is freely accessible here - SOFA, SAPSII

\subsection{Feedbacks of real MIMICIII-OMOP
testing}\label{feedbacks-of-real-mimiciii-omop-testing}

- this work has been done with APHP to test OMOP model in real
statistical condition. A datathon was organised in collaboration with
the MIT.(3) We also test the big data APHP platforms. - most of queries
under 30 second ; simplified model VS MIMIC ; to much normalized for
data scientist)

\subsection{others}\label{others}

\begin{itemize}
\item
  estimation of number of work hours
\item
  ethnicity\_concept\_id : only two strange concept\_name hispanic or
  non\_hispanic
\item
  size of MIMIC OMOP, row number for the bigest relation (measurement)
\item
  \begin{description}
  \item[chartevents and lavents provide many number field as a string
  which is not handy for statistical analyse. We provide a standard and
  easy improval by the community model to extract numerical value from
  string]
  \begin{itemize}
  \tightlist
  \item
    operators have been extracted to fill operator\_concept\_id column
  \item
    numeric value has been extracted to fill value\_as\_number column
  \item
    units of measures have been extracted to fill unit\_concept\_id
    column
  \end{itemize}
  \end{description}
\end{itemize}

\begin{enumerate}
\def\labelenumi{\arabic{enumi}.}
\tightlist
\item
  F. FitzHenry Creating a Common Data Model for Comparative
  Effectiveness with the Observational Medical Outcomes Partnership.
  Appl Clin Inform 2015; 6: 536--547
\item
  \url{https://www.nlm.nih.gov/research/umls/mapping_projects/icd9cm_to_snomedct.html}
\item
  \url{http://blogs.aphp.fr/dat-icu/}
\item
  Y.Dukyong and Al.Conversion and Data Quality Assessment of Electronic
  Health Record Data at a Korean Tertiary Teaching Hospital to a Common
  Data Model for Distributed Network Research.Healthcare Informatics
  Research 2016; 54
\end{enumerate}

\section{Discussion}\label{discussion}

1. OMOP in international context - mapping - concensual guidelines

\begin{enumerate}
\def\labelenumi{\arabic{enumi}.}
\setcounter{enumi}{1}
\item
  \begin{description}
  \item[OMOP in EHR context]
  \begin{itemize}
  \tightlist
  \item
    algorithm real life application
  \item
    juridic access
  \item
    validation algorithm
  \item
    i2b2 (omop) ; FHIR / omop why fhir not replace omop.
  \end{itemize}
  \end{description}
\end{enumerate}

\subsubsection{NOTE\_NLP table}\label{noteux5fnlp-table}

At the time of this study, the table is presented by the OMOP community
as an experimental table. It is clearly a powerful tool and this
evaluation has shown some limitation and some proposal for this. While
the initial idea of the authors clearly indicates that final results
from nlp pipelines only should be stored in the table, we clearly ask
the question wether storing intermediary results in the table such
sections, pos tagging, is of interest for community work. At the very
end, should final information such condition, heart rate extracted from
notes may not be inserted as derived information in dedicated tables
such condition\_occurrence and measurement tables and the note\_nlp
table be used more as a working table for storing and evaluating
intermediary results dedicated to text-miners.

The note\_nlp group encourages to use CDO {[}cite{]} as a standard
terminology to describe notes categories. Instead, we used the OMOP
provided category that is narrower than the MIMIC categorisation.

The note\_nop group encourages to use the LOINC document section from
the CDO list as a standard terminolgy. However, the last release of the
CDO announced they will remove the document sections. Second point is
the omop concept do not release yet the document sections as standards
code. For those reason we decided to wait until a consensual section
terminology emerges before releasing any mapping for this.

\begin{center}\rule{0.5\linewidth}{\linethickness}\end{center}

\textbf{database modelling}

\begin{itemize}
\tightlist
\item
  advantages of normalized
\item
  disadvantages of normalized
\item
  advantages of packed tables
\item
  disadvantages of packed tables
\end{itemize}

- need to propose an optimized version of OMOP -\textgreater{} ==XT: is
that the idea of flattenning the tables to improve query-ability? OMOP
Analytics? technical performances == -\textgreater{} ==NP: indeed,
precalcul most of joins so that queries are focused on one or two table
instead of 5 and more ; in this case, replicating information is not a
problem because it is a frozen dataset and replicated field all come
from one unique field and this is then no error prone by design== ==XT:
What are the pros and cons of this solution against keeping the
structure but producing some ``query plans'' for the frequent accesses?
Several participants at dat-icu told me that at the end of the end, most
of the teams struggled to finally produce queries that were very similar
to their neighbors' queries==

\textbf{terminology mapping}

\begin{itemize}
\tightlist
\item
  athena existing standard mapping
\item
  athena missing concepts
\end{itemize}

What else could have been done instead?

\begin{itemize}
\tightlist
\item
  use of automatic concept mapping processes: but their performances are
  not yet as good as human manual mapping.
\item
  large community mapping: this was too early to open the work to public
  so that effort and direction could be kept
\end{itemize}

OMOP model - enables observational studies to be conducted using
multiple data sources - while confidential personal health data remain
with the original data holders Limitations \#\#\#\#\#\#\#\#\#\#\#\#

\subsubsection{Limitation of our work}\label{limitation-of-our-work}

\begin{itemize}
\tightlist
\item
  mapping coverage
\item
  interpretation of the ETL conventions
\end{itemize}

\subsubsection{Limitation of OMOP}\label{limitation-of-omop}

\begin{itemize}
\tightlist
\item
  structural performances problems
\item
  verbosity of the queries
\item
  structural errors in the model (visit\_detail, note\_nlp)
\item
  investissement needed in the community (forum is very prolific)
\item
  lack of convergence with FHIR
\end{itemize}

\section{Conclusion}\label{conclusion-1}

\begin{itemize}
\tightlist
\item
  time and ressources consuming
\item
  allow worldwide shared algorithm
\end{itemize}
